\documentclass[12pt]{article}

%%%%%%%%%%%%abstract%%%%%%%%%%%%%%%
%Version 1.0 vom 03.11.2021
%Autor: Luis Lütkehellweg
%contact: luis@luetkehellweg.de

%%%%Packages%%%%%
%Document and Language Setup
\usepackage[a4paper,top=2.5cm,bottom=2.5cm,left=3cm,right=2cm,marginparwidth=1.75cm]{geometry}
\usepackage[ngerman]{babel}
\usepackage{fontspec}
\setmainfont{Avenir Next}
%Package for quotation
\usepackage{ktf_cite}

%Bibliothek
\addbibresource{/Users/luisluetkehellweg/Documents/Zitierstile/ktf_cite/sample.bib}

\title{\LaTeX-Cite Style für die KTF der Uni Bonn \\ Dokumentation}
\author{Luis Lütkehellweg}

\begin{document}
\maketitle
\begin{center}
	\large{Version 1.1}
\end{center}


\bigskip

\begin{abstract}
Dieser \LaTeX-Zitierstil setzt die Vorgaben der Katholisch-Theologischen-Fakultät der Universität Bonn um. Diese sind aus dem Reader zum wissenschaftlichen Arbeiten entnommen. Standardmäßig wird ein AuthorTitle-Stil in der Fußnote zum Zitieren genutzt. \bigbreak
\textbf{Release Notes}
\begin{itemize}
	\item URLs werden nun unterstützt und nach den Vorgaben des AT formatiert, das ist insbesondere für das WiBiLex wichtig. Dazu: Lexikoneinträge (\ref{Lexikon}).
\end{itemize}
\end{abstract}
\bigskip
\hrule
\bigskip
\section{Einbinden des Stils}
Für den Stil ist das Package \href{https://github.com/moewew/biblatex-ext}{\textit{biblatex-ext}} erforderlich. Es ist in MiKTeX und TeX Live enthalten.

Um den Stil einzubinden, wird die Datei \verb#ktf_cite.sty# in den gleichen Ordner, wie das \verb#.tex#-Dokument gelegt und kann dann über \verb#\usepackage{ktf_cite}# eingebunden werden. Das Festlegen von Papier, Seitenrändern, etc. erfolgt im \verb#.tex#-Dokument.

Das zugehörige \verb_.bib_-Dokument wird über 
\begin{verbatim}
	\addbibresource{\Path\To\Bib}
\end{verbatim}
eingebunden.

\section{BibTeX-Eingabetypen für die verschiedenen Literaturtypen}
\subsection*{Monographien und Sammelbände}
Für Monographien und das Zitieren ganzer Sammelbände wird der entrytype \nocite{*}
\begin{verbatim}
	@book
\end{verbatim}
benutzt. Der Bibliographieeintrag sieht folgendermaßen aus:
\printbibliography[type=book, title={}]

\subsection*{Aufsatz im Sammelband}
Für Aufsätze in einem Sammelband wird der entrytype
\begin{verbatim}
	@incollection
\end{verbatim}
benutzt. Der Bibliographieeintrag sieht folgendermaßen aus:
\printbibliography[type=incollection, title={}]


\subsection*{Beiträge in Fachzeitschriften}
Für Artikel in Fachzeitschriften wird der entrytype \nocite{*}
\begin{verbatim}
	@article
\end{verbatim}
benutzt. Der Bibliographieeintrag sieht folgendermaßen aus:
\printbibliography[type=article, title={}]

\subsection*{Lexikoneinträge} \label{Lexikon}
Für Lexikoneinträge wird der entrytype \nocite{*}
\begin{verbatim}
	@inreference
\end{verbatim}
benutzt. Onlineressourcen wie das WiBiLex werden auch unterstützt. Dafür müssen die Tags \verb#url# und \verb#urldate# genutzt werden. Das Datum bei \verb#urldate# muss in der Form JJJJ-MM-TT eingetragen werden.   Der Bibliographieeintrag sieht folgendermaßen aus:
\printbibliography[type=inreference, title={}]

\subsection*{Bibelzitate}
Bibelzitate werden noch nicht unterstützt.

\section{Zitieren und Bibliographien}
Standardmäßig nutzt der Stil eine AuthorTitle-Zitierweise in der Fußnote, dafür kann einfach
\begin{verbatim}
	\autocite{cite_key}
\end{verbatim}
benutzt werden. Das sieht beispielsweise so\autocite{berger_liturgie_1990} aus. Alle zitierten Werke tauchen dann automatisch im Literaturverzeichnis auf, das mit 
\begin{verbatim}
	\printbibliography
\end{verbatim}
in das Dokument eingebunden wird.

\section{Melden falscher Zitierweisen und Fehler}
Falls ein Fehler im Stil auffällt, zum Beispiel nicht entsprechend des Readers zitiert wird, kann das mit einer Beispielbibliographie als Issue auf GitHub eingetragen werden oder an \href{mailto:ktf-zitierstil@uni-bonn.de}{ktf-zitierstil@uni-bonn.de} gemeldet werden.

\section{Github Repository}
Das GitHub Repository ist \href{https://github.com/luisluet/ktf_cite/}{hier} verfügbar. Dort finden sich alle Versionen ab 1.1 und die dazugehörigen Release-Notes.

\section{Lizenz und Disclaimer}
Der Stil ist über die MIT License lizensiert. Die Benutzung der Vorlage erfolgt auf eigene Gefahr und unter Ausschluss jeglicher Haftung durch den Ersteller des Stils.
\end{document}